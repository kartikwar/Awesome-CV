%-------------------------------------------------------------------------------
%	SECTION TITLE
%-------------------------------------------------------------------------------
\cvsection{Work Experience}

%-------------------------------------------------------------------------------
%	CONTENT
%-------------------------------------------------------------------------------
\begin{cventries}

%---------------------------------------------------------
    \cventry
    {LEAD DATA SCIENTIST} % Job title
    {Cimpress} % Organization
    {Bengaluru} % Location
    {Sept 2020 - Present} % Date
    {
      \begin{cvitems} % Description(s) of tasks/responsibilities	
      \item{\textbf{Background Removal API} : This API (used in web tool 
      \ulhref{https://backgroundly.io/}{backgroundly}) removes backgrounds from 
      customer-uploaded images and leaves only the salient 
      content in the image. This tool processes approximately 200,000 
      daily requests for custom image engraving services on products 
      like mugs and t-shirts, saving about 3 million USD annually. In this 
      project, I led the development and training of the state-of-theart 
      \ulhref{https://arxiv.org/abs/2003.07711}{fba-matting} algorithm and trained a U-Net model for accurate saliency 
      detection, achieving a visual acceptance rate of 87 percent. 
      Technologies Used: python, pytorch, opencv }
      \item{\textbf{Create Portrait API} : This api imparts a portrait photo effect to
      regular photographs,i.e.,the main subject(s) is maintained in sharp focus, 
      and at the same time, the background is gently blurred to add depth and 
      aesthetic appeal to the photograph. In this project, I combined the 
      open-source mega-depth model with in-house background removal service. 
      Technologies Used: python, pytorch, opencv, pillow }
	\end{cvitems}
    }
%---------------------------------------------------------    
\vspace{2em} % Add extra space between job entries
%---------------------------------------------------------
  \cventry
    {DATA SCIENTIST} % Job title
    {Foghorn Systems} % Organization
    {Pune} % Location
    {Feb 2020 - Sept 2020} % Date
    {
      \begin{cvitems} % Description(s) of tasks/responsibilities	
      \item{\textbf {Mask Detection Model} : This model is responsible for identifying individuals not wearing masks. It was deployed in 
      factories (using IIOT) and real-time alerts were sent when the count exceeded a predefined threshold during the 
      COVID-19 pandemic. I trained a \ulhref{https://roboflow.com/model/mobilenet-ssd-v2}{SSD mobilenet v2} object-detection model 
      and achieved a precision of 85 percent. 
      Technologies Used: python, tensorflow , opencv }
      \item{\textbf{Connector Detection Model}: This model identifies pipe connectors 
      in oil factories, to prevent drilling in unsafe areas. I trained a 
      \ulhref{https://github.com/ultralytics/yolov3}{YOLO v3} object-detection model and achieved a precision of 88 percent. 
      Technologies Used: python, tensorflow , opencv}
	\end{cvitems}
    }
%---------------------------------------------------------    
\vspace{2em} % Add extra space between job entries
%---------------------------------------------------------
  \cventry
    {SENIOR ARTIFICIAL INTELLIGENCE ENGINEER} % Job title
    {Razorthink} % Organization
    {Bengaluru} % Location
    {Sept 2018 - Dec 2019} % Date
    {
      \begin{cvitems} % Description(s) of tasks/responsibilities
        \item{\textbf{Table Detection Model}: This deep learning model is used to 
        detect table like structures in a pdf document.I trained a Faster Rcnn (VGG16) 
        network on pdf documents using the concept of curriculum learning. Achieved a 
        precision of 84 percent. 
        Technologies Used: python, tensorflow, opencv}        
        \item{\textbf{Template Detection Service}: This service compares the layout and 
        structure of two PDF documents.If they are found to be similar,they 
        are categorized under the same class or ’template’. To accomplish this, 
        I built and trained a Siamese network utilizing a pre-trained VGG16 net. 
        Technologies Used: python, tensorflow, opencv, MongoDb}        
	\end{cvitems}
    }
%---------------------------------------------------------    
\vspace{2em} % Add extra space between job entries
%---------------------------------------------------------
  \cventry
    {BACKEND DEVELOPER} % Job title
    {Nowfloats} % Organization
    {Hyderabad} % Location
    {June 2016 - Sept 2018} % Date
    {
      \begin{cvitems} % Description(s) of tasks/responsibilities
        \item {\textbf{Update Synchronize API}: This API synchronizes merchant updates and reviews across 
        various social platforms such as Facebook,LinkedIn, Twitter, and Quikr. The API is used 
        by over 19,000 customers and handles around 50,000 updates on a weekly basis. 
        I made REST APIs, services, lambda functions, cron jobs and created deployement pipeline on 
        ECS. 
        Technologies Used: Python, NodeJs, ECS, Docker, Lambda, Express, MongoDB, Route 53, Ubuntu, SQS, Python}
        \item {\textbf{Update Categorization Service}: This service fetches all the product updates made by the 
        customers on their websites and then categorizes them into either offers, discounts, or sale price. 
        I used natural language processing models such as ‘bag of keywords’ to classify the updates 
        into categories. Technologies Used: Python, Scikit Learn, Pandas, Matplotlib, MongoDB, MySql }
        \item {\textbf{Purchase Probability Model}: This model analyses sales data to predict probability 
        of purchase based on customer characteristics. It helped reduced the customer acquistion cost by 
        50 percent and icreased the conversion rate from under 2 to 20 percent. I trained the classifier using 
        logistic regression and decision trees. 
        Technologies Used: Python, Scikit Learn, Pandas, Matplotlib, MongoDB}
      \end{cvitems}
    }

\end{cventries}