%-------------------------------------------------------------------------------
%	SECTION TITLE
%-------------------------------------------------------------------------------
\cvsection{Work Experience}

%-------------------------------------------------------------------------------
%	CONTENT
%-------------------------------------------------------------------------------
\begin{cventries}

%---------------------------------------------------------
    \cventry
    {LEAD DATA SCIENTIST} % Job title
    {Cimpress} % Organization
    {} % Location
    {Sept 2020 - July 2024} % Date
    {
      \begin{cvitems} % Description(s) of tasks/responsibilities	
      \item{\textbf{Background Removal API} : \newline 
      Directed the development of a high-volume background removal API for \ulhref{https://backgroundly.io/}{backgroundly.io},  
      handling 200,000+ daily requests and saving \$3M annually. \newline 
      Innovated and trained state-of-the-art \ulhref{https://arxiv.org/abs/2003.07711}{FBA-Matting} and 
      U-Net models, achieving an 87 percent visual acceptance rate in saliency 
      detection. \newline 
      Technologies: python, pytorch, opencv }
      \item{\textbf{Create Portrait API} : \newline
      Designed an API to transform regular photos into portrait‐style images by maintaining subject focus and subtly blurring the background.
      \newline
      Collected data for training the model using snowflake and looker. 
      \newline 
      Partnered with cross-functional teams to integrate the \ulhref{https://www.cs.cornell.edu/projects/megadepth/}{MegaDepth model} with in‐house background removal services.
      \newline
      Optimized model inference speed by 20\% using quantization and pruning techniques.
      Technologies: Python, PyTorch, OpenCV, Pillow. }
      \item{\textbf{AI Business-Card Generator} : \newline
      Supervised the development of an AI-powered business card generator leveraging GPT-3.5 to generate layout recommendations and 
      Stable Diffusion for AI-driven background creation, and Retrieval-Augmented Generation (RAG) to utilize a database of business card backgrounds, enabling real-time, print-ready customization and reducing design iteration time by 60\%. 
      \newline
      Deployed ML pipelines with AWS Lambda \& CI/CD using GitHub Actions, automating model updates.
      \newline
      Technologies: Python, OpenAI API, Hugging Face, Stable Diffusion, FastAPI, AWS Lambda, Docker.}
	\end{cvitems}
    }
%---------------------------------------------------------    
\vspace{2em} % Add extra space between job entries
%---------------------------------------------------------
  \cventry
    {DATA SCIENTIST} % Job title
    {Foghorn Systems} % Organization
    {} % Location
    {Feb 2020 - Sept 2020} % Date
    {
      \begin{cvitems} % Description(s) of tasks/responsibilities	
        \item{\textbf{Mask Detection Model}: \newline 
        Pioneered a real-time mask detection system to identify individuals not wearing masks in factories during the COVID-19 pandemic, triggering 
        real-time alerts when counts exceeded thresholds via IIoT. \newline
        Trained an \ulhref{https://roboflow.com/model/mobilenet-ssd-v2}{SSD MobileNet V2} object-detection model, 
        achieving 85\% precision. \newline Technologies: Python, TensorFlow, OpenCV.}        
        \item{\textbf{Connector Detection Model}: \newline Architected a pipeline for identifying unsafe pipe connectors in oil factories, preventing 
        hazardous drilling operations. \newline
        Elevated model precision to 88\% using a \ulhref{https://github.com/ultralytics/yolov3}{YOLO v3} object-detection framework. \newline
        Technologies: Python, TensorFlow, OpenCV.}
        \item{\textbf{Predictive Maintenance Model}: \newline Developed a time series forecasting model to predict equipment failures using sensor telemetry data from industrial machines. \newline
        This helped in reducing unplanned downtime by 25 percent. \newline 
        Technologies: Python, TensorFlow, OpenCV.}
	\end{cvitems}
    }
%---------------------------------------------------------    
\vspace{2em} % Add extra space between job entries
%---------------------------------------------------------
  \cventry
    {SENIOR ARTIFICIAL INTELLIGENCE ENGINEER} % Job title
    {Razorthink} % Organization
    {} % Location
    {Sept 2018 - Dec 2019} % Date
    {
      \begin{cvitems} % Description(s) of tasks/responsibilities
        \item{\textbf{Table Detection Model}: Formulated a deep learning model to 
        detect table-like structures in PDF documents. \newline Trained a Faster 
        R-CNN (VGG16) network using curriculum learning, 
        achieving 84\% precision. \newline Technologies: Python, TensorFlow, OpenCV.}      
        \item{\textbf{Template Detection Service}: Engineered an AI-powered service to compare the 
        layout and structure of PDF documents, classifying similar documents under 
        the same template. \newline Oversaw the development and deployment of a Siamese network using a pre-trained 
        VGG16 model. \newline Technologies: Python, TensorFlow, OpenCV, MongoDB.}       
	\end{cvitems}
    }
%---------------------------------------------------------    
\vspace{2em} % Add extra space between job entries
%---------------------------------------------------------
  \cventry
    {BACKEND DEVELOPER} % Job title
    {Nowfloats} % Organization
    {} % Location
    {June 2016 - Sept 2018} % Date
    {
      \begin{cvitems} % Description(s) of tasks/responsibilities
        \item{\textbf{Update Synchronize API}: Orchestrated an API to synchronize merchant updates and reviews across social platforms like Facebook, LinkedIn, Twitter, and Quikr, serving over 19,000 customers and processing 50,000 weekly updates. \newline Implemented REST APIs, services, Lambda functions, cron jobs, and created deployment pipelines on ECS. \newline Technologies: Python, NodeJS, ECS, Docker, Lambda, Express, MongoDB, Route 53, Ubuntu, SQS.}
        \item{\textbf{Update Categorization Service}: Collaborated on an NLP-based service that fetched and categorized customer product updates into offers, discounts, or sale prices using natural language processing algorithms such as ‘bag of words’ and an SVC (Support Vector Classifier). \newline 
        Handled class imbalance between updates by using appropriate metrics like f1-score instead of accuracy. 
        Technologies: Python, Scikit-Learn, Pandas, Matplotlib, MongoDB, MySQL.}
        \item{\textbf{Purchase Probability Model}: Conceptualized a predictive model to analyze sales data and predict purchase probabilities based on customer characteristics, reducing acquisition costs by 50\% and increasing conversion rates from under 2\% to 20\%. \newline
        Performed feature engineering and exploratory data analysis to find the right features for classification.
        \newline Developed classifiers using logistic regression and decision trees. \newline Technologies: Python, Scikit-Learn, Pandas, Matplotlib, MongoDB.}
      \end{cvitems}
    }

\end{cventries}